\HeaderA{foldchange}{Compute fold-change or convert between log-ratio and fold-change.}{foldchange}
\aliasA{foldchange2logratio}{foldchange}{foldchange2logratio}
\aliasA{logratio2foldchange}{foldchange}{logratio2foldchange}
\keyword{math}{foldchange}
\begin{Description}\relax
\code{foldchange} computes the fold change for two sets of values.
\code{logratio2foldchange} converts values from log-ratios to fold
changes.  \code{foldchange2logratio} does the reverse.
\end{Description}
\begin{Usage}
\begin{verbatim}
foldchange(num,denom)
logratio2foldchange(logratio, base=2)
foldchange2logratio(foldchange, base=2)
\end{verbatim}
\end{Usage}
\begin{Arguments}
\begin{ldescription}
\item[\code{num,denom}] vector/matrix of numeric values
\item[\code{logratio}] vector/matrix of log-ratio values
\item[\code{foldchange}] vector/matrix of fold-change values
\item[\code{base}] Exponential base for the log-ratio.
\end{ldescription}
\end{Arguments}
\begin{Details}\relax
Fold changes are commonly used in the biological sciences as a
mechanism for comparing the relative size of two measurements.  They
are computed as: \eqn{\frac{num}{denom}}{num/denom} if
\eqn{num>denom}{}, and as \eqn{\frac{-denom}{num}}{-denom/num}
otherwise.

Fold-changes have the advantage of ease of interpretation and symmetry
about \eqn{num=denom}{}, but suffer from a discontinuty between -1 and
1, which can cause significant problems when performing data
analysis.  Consequently statisticians prefer to use log-ratios.
\end{Details}
\begin{Value}
A vector or matrix of the same dimensions as the input containing the
converted values.
\end{Value}
\begin{Author}\relax
Gregory R. Warnes \email{warnes@bst.rochester.edu}
\end{Author}
\begin{Examples}
\begin{ExampleCode}

  a <- 1:21
  b <- 21:1

  f <- foldchange(a,b)

  cbind(a,b,f)

\end{ExampleCode}
\end{Examples}

