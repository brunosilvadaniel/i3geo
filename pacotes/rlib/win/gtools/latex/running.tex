\HeaderA{running}{Apply a Function Over Adjacent Subsets of a Vector}{running}
\keyword{misc}{running}
\begin{Description}\relax
Applies a function over subsets of the vector(s) formed by
taking a fixed number of previous points.
\end{Description}
\begin{Usage}
\begin{verbatim}
running(X, Y=NULL, fun=mean, width=min(length(X), 20),
        allow.fewer=FALSE, pad=FALSE, align=c("right", "center","left"),
        simplify=TRUE, by, ...)
\end{verbatim}
\end{Usage}
\begin{Arguments}
\begin{ldescription}
\item[\code{X}] data vector 
\item[\code{Y}] data vector (optional) 
\item[\code{fun}] Function to apply. Default is \code{mean}
\item[\code{width}] Integer giving the number of vector elements to include
in the subsets.  Defaults to the lesser of the length of the data and
20 elements.
\item[\code{allow.fewer}] Boolean indicating whether the function should be
computed for subsets with fewer than \code{width} points
\item[\code{pad}] Boolean indicating whether the returned results should
be 'padded' with NAs corresponding to sets with less than
\code{width} elements.  This only applies when when
\code{allow.fewer} is FALSE.
\item[\code{align}] One of "right", "center", or "left".
This controls the relative location of `short' subsets with less
then \code{width} elements: "right" allows short subsets only at the
beginning of the sequence so that all of the complete subsets are at
the end of the sequence (i.e. `right aligned'), "left" allows short
subsets only at the end of the data so that the complete subsets
are `left aligned', and "center" allows short subsets at both ends
of the data so that complete subsets are `centered'.

\item[\code{simplify}] Boolean.  If FALSE the returned object will be a list
containing one element per evaluation.  If TRUE, the returned
object will be coerced into a vector (if the computation returns a
scalar) or a matrix (if the computation returns multiple values).
Defaults to FALSE.
\item[\code{by}] Integer separation between groups. If \code{by=width} will
give non-overlapping windows. Default is missing, in which case
groups will start at each value in the X/Y range.
\item[\code{...}] parameters to be passed to \code{fun} 
\end{ldescription}
\end{Arguments}
\begin{Details}\relax
\code{running} applies the specified function to
a sequential windows on \code{X} and (optionally) \code{Y}.  If
\code{Y} is specified the function must be bivariate.
\end{Details}
\begin{Value}
List (if \code{simplify==TRUE}), vector, or matrix containg the
results of applying the function \code{fun} to the
subsets of \code{X} (\code{running}) or \code{X} and \code{Y}.

Note that this function will create a vector or matrix even for
objects which are not simplified by \code{sapply}.
\end{Value}
\begin{Author}\relax
Gregory R. Warnes \email{warnes@bst.rochester.edu},
with contributions by Nitin Jain \email{nitin.jain@pfizer.com}.
\end{Author}
\begin{SeeAlso}\relax
\code{\LinkA{wapply}{wapply}} to apply a function over an x-y window
centered at each x point, \code{\LinkA{sapply}{sapply}},
\code{\LinkA{lapply}{lapply}}
\end{SeeAlso}
\begin{Examples}
\begin{ExampleCode}

# show effect of pad
running(1:20, width=5)
running(1:20, width=5, pad=TRUE)

# show effect of align
running(1:20, width=5, align="left", pad=TRUE)
running(1:20, width=5, align="center", pad=TRUE)
running(1:20, width=5, align="right", pad=TRUE)

# show effect of simplify
running(1:20, width=5, fun=function(x) x )  # matrix
running(1:20, width=5, fun=function(x) x, simplify=FALSE) # list

# show effect of by
running(1:20, width=5)       # normal
running(1:20, width=5, by=5) # non-overlapping
running(1:20, width=5, by=2) # starting every 2nd

# Use 'pad' to ensure correct length of vector, also show the effect
# of allow.fewer.
par(mfrow=c(2,1))
plot(1:20, running(1:20, width=5, allow.fewer=FALSE, pad=TRUE), type="b")
plot(1:20, running(1:20, width=5, allow.fewer=TRUE,  pad=TRUE), type="b")
par(mfrow=c(1,1))

# plot running mean and central 2 standard deviation range
# estimated by *last* 40 observations
dat <- rnorm(500, sd=1 + (1:500)/500 )
plot(dat)
sdfun <- function(x,sign=1) mean(x) + sign * sqrt(var(x))
lines(running(dat, width=51, pad=TRUE, fun=mean), col="blue")
lines(running(dat, width=51, pad=TRUE, fun=sdfun, sign=-1), col="red")
lines(running(dat, width=51, pad=TRUE, fun=sdfun, sign= 1), col="red")

# plot running correlation estimated by last 40 observations (red)
# against the true local correlation (blue)
sd.Y <- seq(0,1,length=500)

X <- rnorm(500, sd=1)
Y <- rnorm(500, sd=sd.Y)

plot(running(X,X+Y,width=20,fun=cor,pad=TRUE),col="red",type="s")

r <- 1 / sqrt(1 + sd.Y^2) # true cor of (X,X+Y)
lines(r,type="l",col="blue")
\end{ExampleCode}
\end{Examples}

