\HeaderA{plot.deldir}{Produce a plot of the Delaunay triangulation and Dirichlet (Voronoi)
tesselation of a planar point set, as constructed by the function deldir.}{plot.deldir}
\begin{Description}\relax
This is a method for plot.
\end{Description}
\begin{Usage}
\begin{verbatim}
## S3 method for class 'deldir':
plot(x,add=FALSE,wlines=c('both','triang','tess'),
                      wpoints=c('both','real','dummy','none'),
                      number=FALSE,cex=1,nex=1,col=NULL,lty=NULL,
                      pch=NULL,xlim=NULL,ylim=NULL,xlab='x',ylab='y',...)

\end{verbatim}
\end{Usage}
\begin{Arguments}
\begin{ldescription}
\item[\code{x}] An object of class "deldir" as constructed by the function deldir.

\item[\code{add}] logical argument; should the plot be added to an existing plot?

\item[\code{wlines}] "which lines?".  I.e.  should the Delaunay triangulation be plotted
(wlines='triang'), should the Dirichlet tessellation be plotted
(wlines='tess'), or should both be plotted (wlines='both', the
default) ?

\item[\code{wpoints}] "which points?".  I.e.  should the real points be plotted
(wpoints='real'), should the dummy points be plotted
(wpoints='dummy'), should both be plotted (wpoints='both', the
default) or should no points be plotted (wpoints='none')?

\item[\code{number}] Logical argument, defaulting to \code{FALSE}; if \code{TRUE} then the
points plotted will be labelled with their index numbers
(corresponding to the row numbers of the matrix "summary" in the
output of deldir).

\item[\code{cex}] The value of the character expansion argument cex to be used
with the plotting symbols for plotting the points.

\item[\code{nex}] The value of the character expansion argument cex to be used by the
text function when numbering the points with their indices.  Used only
if number=\code{TRUE}.

\item[\code{col}] the colour numbers for plotting the triangulation, the tesselation,
the data points, the dummy points, and the point numbers, in that
order; defaults to c(1,1,1,1,1).  If fewer than five numbers are
given, they are recycled.  (If more than five numbers are given, the
redundant ones are ignored.)

\item[\code{lty}] the line type numbers for plotting the triangulation and the
tesselation, in that order; defaults to 1:2.  If only one value is
given it is repeated.  (If more than two numbers are given, the
redundant ones are ignored.)

\item[\code{pch}] the plotting symbols for plotting the data points and the dummy
points, in that order; may be either integer or character; defaults
to 1:2.  If only one value is given it is repeated.  (If more than
two values are given, the redundant ones are ignored.)

\item[\code{xlim}] the limits on the x-axis.  Defaults to rw[1:2] where rw is the
rectangular window specification returned by deldir().

\item[\code{ylim}] the limits on the y-axis.  Defaults to rw[3:4] where rw is the
rectangular window specification returned by deldir().

\item[\code{xlab}] label for the x-axis.  Defaults to \code{x}.  Ignored if
\code{add=TRUE}.

\item[\code{ylab}] label for the y-axis.  Defaults to \code{y}.  Ignored if
\code{add=TRUE}.

\item[\code{...}] Further plotting parameters to be passed to \code{plot()}
\code{segments()} or \code{points()}.  Unlikely to be used.

\end{ldescription}
\end{Arguments}
\begin{Details}\relax
The points in the set being triangulated are plotted with distinguishing
symbols.  By default the real points are plotted as circles (pch=1) and the
dummy points are plotted as triangles (pch=2).
\end{Details}
\begin{Section}{Side Effects}
A plot of the points being triangulated is produced or added to
an existing plot.  As well, the edges of the Delaunay
triangles and/or of the Dirichlet tiles are plotted.  By default
the triangles are plotted with solid lines (lty=1) and the tiles
with dotted lines (lty=2).
\end{Section}
\begin{Author}\relax
Rolf Turner
\email{r.turner@auckland.ac.nz}
\url{http://www.math.unb.ca/~rolf}
\end{Author}
\begin{SeeAlso}\relax
\code{\LinkA{deldir}{deldir}()}
\end{SeeAlso}
\begin{Examples}
\begin{ExampleCode}
## Not run: 
try <- deldir(x,y,list(ndx=2,ndy=2),c(0,10,0,10))
plot(try)
#
deldir(x,y,list(ndx=4,ndy=4),plot=TRUE,add=TRUE,wl='te',
       col=c(1,1,2,3,4),num=TRUE)
# Plots the tesselation, but does not save the results.
try <- deldir(x,y,list(ndx=2,ndy=2),c(0,10,0,10),plot=TRUE,wl='tr',
              wp='n')
# Plots the triangulation, but not the points, and saves the returned
structure.
## End(Not run)
\end{ExampleCode}
\end{Examples}

