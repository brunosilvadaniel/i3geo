\HeaderA{assert}{Generate an error if an expression is not true.}{assert}
\keyword{programming}{assert}
\begin{Description}\relax
Generate an error if an expression is not true.
\end{Description}
\begin{Usage}
\begin{verbatim}
assert(FLAG)
\end{verbatim}
\end{Usage}
\begin{Arguments}
\begin{ldescription}
\item[\code{FLAG}] Expression that should evaluate to a boolean vector
\end{ldescription}
\end{Arguments}
\begin{Details}\relax
Assert generate an error if its aregument does not evaluate to 
boolean (vector) containing only \code{TRUE} values.  This is useful
for defensinve programming as it provides a mechanism for checking
that certain facts, the 'assertions', do in fact hold.  Checking of 
'assertions' is an important tool in the development of robust program
code.
\end{Details}
\begin{Value}
None.  Evaluated only for its side effect.
\end{Value}
\begin{Author}\relax
Gregory R. Warnes \email{warnes@bst.rochester.edu}
\end{Author}
\begin{SeeAlso}\relax
\code{\LinkA{stop}{stop}}, \code{\LinkA{warning}{warning}}
\end{SeeAlso}
\begin{Examples}
\begin{ExampleCode}

## Trivial example
posSqrt <- function(x)
  {
    assert(x>=0)
    sqrt(x)
  }

posSqrt(1:10) # works fine, no messages
## Not run: 
posSqrt(-5:5) # generates an error, since the asssertion is not met
## End(Not run)

\end{ExampleCode}
\end{Examples}

