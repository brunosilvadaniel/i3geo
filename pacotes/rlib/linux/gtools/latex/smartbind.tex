\HeaderA{smartbind}{Efficient rbind of data framesy, even if the column names don't match}{smartbind}
\keyword{manip}{smartbind}
\begin{Description}\relax
Efficient rbind of data frames, even if the column names don't match
\end{Description}
\begin{Usage}
\begin{verbatim}
smartbind(...)
\end{verbatim}
\end{Usage}
\begin{Arguments}
\begin{ldescription}
\item[\code{...}] Data frames to combine
\end{ldescription}
\end{Arguments}
\begin{Value}
The returned data frame will contain:
\begin{ldescription}
\item[\code{columns}] all columns present in any provided data frame
\item[\code{rows}] a set of rows from each provided data frame, with values
in columns not present in the given data frame filled with missing
(\code{NA}) values.
\end{ldescription}

The data type of columns will be preserved, as long as all data frames
with a given column name agree on the data type of that column.  If
the data frames disagree, the column will be converted into a
character strings.  The user will need to coerce such character
columns into an appropriate type.
\end{Value}
\begin{Author}\relax
Gregory R. Warnes \email{warnes@bst.rochester.edu}
\end{Author}
\begin{SeeAlso}\relax
\code{\LinkA{rbind}{rbind}}, \code{\LinkA{cbind}{cbind}}
\end{SeeAlso}
\begin{Examples}
\begin{ExampleCode}

  df1 <- data.frame(A=1:10, B=LETTERS[1:10], C=rnorm(10) )
  df2 <- data.frame(A=11:20, D=rnorm(10), E=letters[1:10] )

  # rbind would fail
## Not run: 
  rbind(df1, df2)
  # Error in match.names(clabs, names(xi)) : names do not match previous
  # names:
  #     D, E
## End(Not run)
  # but smartbind combines them, appropriately creating NA entries
  smartbind(df1, df2)


\end{ExampleCode}
\end{Examples}

