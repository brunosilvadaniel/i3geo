\HeaderA{invalid}{Test if a value is missing, empty, or contains only NA or NULL values}{invalid}
\keyword{programming}{invalid}
\begin{Description}\relax
Test if a value is missing, empty, or contains only NA or NULL values.
\end{Description}
\begin{Usage}
\begin{verbatim}
invalid(x)
\end{verbatim}
\end{Usage}
\begin{Arguments}
\begin{ldescription}
\item[\code{x}] value to be tested
\end{ldescription}
\end{Arguments}
\begin{Value}
Logical value.
\end{Value}
\begin{Author}\relax
Gregory R. Warnes \email{warnes@bst.rochester.edu}
\end{Author}
\begin{SeeAlso}\relax
\code{\LinkA{missing}{missing}}, \code{\LinkA{is.na}{is.na}},
\code{\LinkA{is.null}{is.null}}
\end{SeeAlso}
\begin{Examples}
\begin{ExampleCode}

invalid(NA)
invalid()
invalid(c(NA,NA,NULL,NA))

invalid(list(a=1,b=NULL))

# example use in a function
myplot <- function(x,y) {
                if(invalid(y)) {
                        y <- x
                        x <- 1:length(y)
                }
                plot(x,y)
        }
myplot(1:10)
myplot(1:10,NA)
\end{ExampleCode}
\end{Examples}

