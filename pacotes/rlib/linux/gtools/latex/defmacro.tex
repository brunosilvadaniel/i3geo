\HeaderA{defmacro}{Define a macro}{defmacro}
\aliasA{strmacro}{defmacro}{strmacro}
\keyword{programming}{defmacro}
\begin{Description}\relax
\code{defmacro} define a macro that uses R expression replacement

\code{strmacro} define a macro that uses string replacement
\end{Description}
\begin{Usage}
\begin{verbatim}
defmacro(..., expr)
strmacro(..., expr, strexpr)
\end{verbatim}
\end{Usage}
\begin{Arguments}
\begin{ldescription}
\item[\code{...}] macro argument list 
\item[\code{expr}] R expression defining the macro body 
\item[\code{strexpr}] character string defining the macro body 
\end{ldescription}
\end{Arguments}
\begin{Details}\relax
\code{defmacro} and \code{strmacro} create a macro from the expression
given in \code{expr}, with formal arguments given by the other
elements of the argument list.

A macro is similar to a function definition, the arguments are
handled.  In a function, formal arguments are simply variables that 
contains the result of evaluating the expressions provided to the
function call.  In contrast, macros actually modify the macro body by
\code{replacing} each formal argument by the expression
(\code{defmacro}) or string (\code{strmacro}) provided to the macro
call.

For \code{defmacro}, the special argument name \code{DOTS} will be
replaced by \code{...} in the formal argument list of the macro so
that \code{...}  in the body of the expression can be used to obtain
any additional arguments passed to the macro. For \code{strmacro} you
can mimic this behavior providing a \code{DOTS=""} argument.  This is
illustrated by the last example below.

Macros are often useful for creating new functions during code execution.
\end{Details}
\begin{Value}
A macro function.
\end{Value}
\begin{Note}\relax
Note that because [the defmacro code] works on the parsed expression,
not on a text string, defmacro avoids some of the problems of
traditional string substitution macros such as \code{strmacro} and the C
preprocessor macros. For example, in
\begin{alltt}
  mul <- defmacro(a, b, expr=\{a*b\})
\end{alltt}
a C programmer might expect
\code{mul(i, j + k)} to expand (incorrectly) to \code{i*j + k}. In fact it
expands correctly, to the equivalent of \code{i*(j + k)}.

For a discussion of the differences between functions
and macros, please Thomas Lumley's R-News article (reference below).
\end{Note}
\begin{Author}\relax
Thomas Lumley wrote \code{defmacro}.  Gregory R. Warnes
\email{warnes@bst.rochester.edu} enhanced it and created
\code{strmacro}.
\end{Author}
\begin{References}\relax
The original \code{defmacro} code was directly taken from:

Lumley T. "Programmer's Niche: Macros in {R}", R News, 2001, Vol 1,
No. 3, pp 11--13, \url{http://CRAN.R-project.org/doc/Rnews/}
\end{References}
\begin{SeeAlso}\relax
\code{\LinkA{function}{function}}
\code{\LinkA{substitute}{substitute}},
\code{\LinkA{eval}{eval}},
\code{\LinkA{parse}{parse}},
\code{\LinkA{source}{source}},
\code{\LinkA{parse}{parse}},
\end{SeeAlso}
\begin{Examples}
\begin{ExampleCode}
####
# macro for replacing a specified missing value indicator with NA
# within a dataframe
###
setNA <- defmacro(df, var, values,
                  expr={
                         df$var[df$var %in% values] <- NA
                       })

# create example data using 999 as a missing value indicator
d <- data.frame(
   Grp=c("Trt", "Ctl", "Ctl", "Trt", "Ctl", "Ctl", "Trt", "Ctl", "Trt", "Ctl"),
   V1=c(1, 2, 3, 4, 5, 6, 999, 8,   9,  10),
   V2=c(1, 1, 1, 1, 1, 2, 999, 2, 999, 999)
               )
d

# Try it out
setNA(d, V1, 999)
setNA(d, V2, 999)
d

###
# Expression macro
###
plot.d <- defmacro( df, var, DOTS, col="red", title="", expr=
  plot( df$var ~ df$Grp, type="b", col=col, main=title, ... )
)

plot.d( d, V1)
plot.d( d, V1, col="blue" )
plot.d( d, V1, lwd=4)  # use optional 'DOTS' argument

###
# String macro (note the quoted text in the calls below)
# 
# This style of macro can be useful when you are reading
# function arguments from a text file
###
plot.s <- strmacro( DF, VAR, COL="'red'", TITLE="''", DOTS="", expr=
  plot( DF$VAR ~ DF$Grp, type="b", col=COL, main=TITLE, DOTS)
)

plot.s( "d", "V1")
plot.s( DF="d", VAR="V1", COL='"blue"' ) 
plot.s( "d", "V1", DOTS='lwd=4')  # use optional 'DOTS' argument


#######
# Create a macro that defines new functions
######
plot.sf <- defmacro(type='b', col='black',
                    title=deparse(substitute(x)), DOTS, expr=
  function(x,y) plot( x,y, type=type, col=col, main=title, ...)
)

plot.red  <- plot.sf(col='red',title='Red is more Fun!')
plot.blue <- plot.sf(col='blue',title="Blue is Best!", lty=2)

plot.red(1:100,rnorm(100))
plot.blue(1:100,rnorm(100))

\end{ExampleCode}
\end{Examples}

